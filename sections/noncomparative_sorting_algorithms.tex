\documentclass[../main.tex]{subfiles}

\subsection*{Counting Sort}

Nos va a permitir ordenar datos numéricos discretos que estén en un rango acotado y conocido (se debe poder obtener fácil el mínimo y el máximo). 

\lstinputlisting[language=Python]{algorithms/counting_sort.py}

\subsection*{Radix Sort}

\begin{itemize}
    \item Se usa cuando queremos ordenar cosas por distintos criterios.
    \item Trabaja con elementos a ordenar que tengan varios dígitos o componentes.
    \item Utiliza un ordenamiento auxiliar que \textbf{tiene que ser estable}. Idealmente, que sea lineal.
    \item Cada elemento tiene que tener la misma cantidad de cifras, o muy similar.
    \item Ordena (utilizando el arreglo auxiliar) de la cifra \textbf{menos significativa} a la cifra \textbf{más significativa}.
\end{itemize}

Necesitamos que cada uno de estos dígitos, cifras o componentes, se ordene internamente con un ordenamiento estable auxiliar (si no es estable entonces no va a funcionar) generalmente \textit{counting sort}.

Sirve para números que estén en cualquier base, arreglos, cadenas y cualquier cosa que tenga varias cifras de distinto valor. Pero todos los elementos deben tener la misma cantidad de cifras.

\lstinputlisting[language=Python]{algorithms/radix_sort.py}

\subsection*{Bucket Sort}

\begin{itemize}
    \item En este caso queremos ordenar algo que puede no tener un rango enumerable (discreto).
    \item Debe ser conocida la distribución de los datos.
    \item Los datos debe n ser uniformemente distribuidos.
    \item Bastante útil si no podemos aplicar \textit{counting sort} o \textit{radix sort} (ej: números decimales).
\end{itemize}

Vamos a querer ordenar algo que no tenga un rango enumerable, pero algo que si tenga una distribución conocida. Inicialmente lo que queremos es que esté uniformemente distribuido. A veces no vamos a tener eso pero si vamos a saber como está distribuido y lo vamos a poder uniformizar. Por ejemplo podemos ordenar un arreglo de números con decimales infinitos que sabemos que están uniformemente distribuidos (con radix y counting sort no podemos).

\lstinputlisting[language=Python]{algorithms/bucket_sort.py}