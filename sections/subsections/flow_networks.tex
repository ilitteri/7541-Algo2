\documentclass[../main.tex]{subfiles}

\subsubsection{Restricciones}
\begin{itemize}
    \item No se aceptan bucles.
    \item No pueden haber ciclos de dos vértices (aristas antiparalelas).
    \item Todos los vértices pueden llegar al sumidero de alguna forma.
    \item Solo hay una fuente.
    \item Solo hay un sumidero.
\end{itemize}

\subsubsection{Red Residual}

Es el grafo con los mismos vértices, pero tiene como aristas:
\begin{enumerate}
    \item Las mismas del original, al que aún les queda capacidad para utilizar. El peso es esa capacidad restante.
    \item La arista opuesta, con peso la capacidad utilizada.
    \item Si alguno ed os anteriores es 0, no hay arista.
\end{enumerate}

En el peor de los casos el grafo residual tiene el doble de aristas.

Un grafo residual es una red de flujo, salvo por las aristas antiparalelas.

\subsubsection{Camino de aumento (\textit{augmenting path})}

Si encontramos un camino de la fuente (s) al sumidero (t) en el grafo residual, entonces encontramos un camino por el que podemos aumentar el flujo.

Buscamos aumentar el flujo total, pero puede reducirse el que paso por una arista en particular (para el mismo fin).