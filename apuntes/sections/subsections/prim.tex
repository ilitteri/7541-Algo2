\documentclass[../../main.tex]{subfiles}

\subsubsection{Paso a paso}
\begin{enumerate}
    \item Comienza en un vértice aleatorio, encola en un heap todas sus aristas.
    \item El vértice origen queda como visitado.
    \item Mientras el heap no está vacío, saca una arista.
    \item Si ambos vértices de la arista fueron visitados, se descarta la arista (formaría un ciclo).
    \item Caso contrario, se agrega la arista al árbol, y se encolan todas las aristas del nuevo vértice visitado.
    \item Volvemos al paso 3.
\end{enumerate}

\subsubsection{Algoritmo}
\lstinputlisting[language=Python]{algorithms/graphs/prim.py}

\subsubsection{Complejidad}
\begin{itemize}
    \item Obtener el vértice aleatorio es $\Theta(1)$, $\Theta(V)$ en el peor de los casos.
    \item Crear el árbol a partir de los vértices deñ grafo original es $\Theta(V)$.
    \item Ver los adyacentes de un vértice en particular $\Theta(1)$ (despreciable).
    \item En el peor de los casos, encolamos y desencolamos todas las aristas del grafo. Entonces, costará $\Theta(E\cdot \log(V))$.
\end{itemize}
