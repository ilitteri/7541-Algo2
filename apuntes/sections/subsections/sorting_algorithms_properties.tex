\documentclass[../main.tex]{subfiles}

\subsubsection{Estabilidad en algoritmos de ordenamiento}

En un algoritmo de ordenamiento estable, los elementos que coinciden en su clave de ordenamiento aparecen, en el arreglo de salida, en el mismo orden relativo que en el arreglo original.

Un algoritmo de ordenamiento es estable cuando se asegura que el orden relativo de los elementos de \textit{misma clave} (que son iguales para el ordenamiento) es idéntico a la salida que a la entrada.

\subsubsection{Algoritmos de ordenamiento \textit{in place}}

Un algoritmo de ordenamiento es In-Place si ordena los elementos sobre el arreglo original. Un ejemplo de un algoritmo que no lo es, es \textit{Merge Sort} ya que éste realiza una copia entera del arreglo para realizar el \textit{merge}.

Un algoritmo de ordenamiento es in-place cuando ordena directamente sobre el arreglo original (utiliza $\Omega(1)$ de espacio adicional).